\lstset{
	language=HTML,
	tabsize=3,
	numbers=left,
	stepnumber=1,    
	firstnumber=1,
	numberfirstline=true,
	postbreak=\raisebox{0ex}[0ex][0ex]{\ensuremath{\color{red}\hookrightarrow\space}},
	xleftmargin=1.5cm,
	xrightmargin=1.5cm,
	frame=shadowbox,
	rulesepcolor=\color{black},
	captionpos=b,
	abovecaptionskip=12pt,
	framextopmargin=0.2cm,
	framexbottommargin=0.2cm,
	framexleftmargin=0.8cm,
	framexrightmargin=0.8cm,
	    aboveskip=2.5em,
	    belowskip=2.5em,
	numbersep=0.3cm,
	breaklines=true,columns=fullflexible,
	basicstyle=\small\normalfont,
	extendedchars=true,
	literate=
	{á}{{\'a}}1 {é}{{\'e}}1 {í}{{\'i}}1 {ó}{{\'o}}1 {ú}{{\'u}}1
	{Á}{{\'A}}1 {É}{{\'E}}1 {Í}{{\'I}}1 {Ó}{{\'O}}1 {Ú}{{\'U}}1
}

\renewcommand{\lstlistingname}{Fragmento}

%\lstset{basicstyle=\footnotesize\ttfamily}

% Para que funcione el multiples rangos dentro de un listing
\makeatletter
\lst@Key{numbers}{none}{%
    \let\lst@PlaceNumber\@empty
    \lstKV@SwitchCases{#1}%
    {none&\\%
     left&\def\lst@PlaceNumber{\llap{\normalfont
                \lst@numberstyle{\thelstnumber}\kern\lst@numbersep}}\\%
     leftliteral&\def\lst@PlaceNumber{\llap{\normalfont
                \lst@numberstyle{\the\lst@lineno}\kern\lst@numbersep}}\\%
     right&\def\lst@PlaceNumber{\rlap{\normalfont
                \kern\linewidth \kern\lst@numbersep
                \lst@numberstyle{\thelstnumber}}}%
    }{\PackageError{Listings}{Numbers #1 unknown}\@ehc}}
\makeatother

% https://tex.stackexchange.com/questions/110187/listings-line-numbers-that-match-the-linerange-specification

% \makeatletter
% \lst@Key{matchrangestart}{f}{\lstKV@SetIf{#1}\lst@ifmatchrangestart}
% \def\lst@SkipToFirst{%
%     \lst@ifmatchrangestart\c@lstnumber=\numexpr-1+\lst@firstline\fi
%     \ifnum \lst@lineno<\lst@firstline
%         \def\lst@next{\lst@BeginDropInput\lst@Pmode
%         \lst@Let{13}\lst@MSkipToFirst
%         \lst@Let{10}\lst@MSkipToFirst}%
%         \expandafter\lst@next
%     \else
%         \expandafter\lst@BOLGobble
%     \fi}
% \makeatother

% matchrangestart=t